% Options for packages loaded elsewhere
\PassOptionsToPackage{unicode}{hyperref}
\PassOptionsToPackage{hyphens}{url}
%
\documentclass[
]{article}
\usepackage{lmodern}
\usepackage{amssymb,amsmath}
\usepackage{ifxetex,ifluatex}
\ifnum 0\ifxetex 1\fi\ifluatex 1\fi=0 % if pdftex
  \usepackage[T1]{fontenc}
  \usepackage[utf8]{inputenc}
  \usepackage{textcomp} % provide euro and other symbols
\else % if luatex or xetex
  \usepackage{unicode-math}
  \defaultfontfeatures{Scale=MatchLowercase}
  \defaultfontfeatures[\rmfamily]{Ligatures=TeX,Scale=1}
\fi
% Use upquote if available, for straight quotes in verbatim environments
\IfFileExists{upquote.sty}{\usepackage{upquote}}{}
\IfFileExists{microtype.sty}{% use microtype if available
  \usepackage[]{microtype}
  \UseMicrotypeSet[protrusion]{basicmath} % disable protrusion for tt fonts
}{}
\makeatletter
\@ifundefined{KOMAClassName}{% if non-KOMA class
  \IfFileExists{parskip.sty}{%
    \usepackage{parskip}
  }{% else
    \setlength{\parindent}{0pt}
    \setlength{\parskip}{6pt plus 2pt minus 1pt}}
}{% if KOMA class
  \KOMAoptions{parskip=half}}
\makeatother
\usepackage{xcolor}
\IfFileExists{xurl.sty}{\usepackage{xurl}}{} % add URL line breaks if available
\IfFileExists{bookmark.sty}{\usepackage{bookmark}}{\usepackage{hyperref}}
\hypersetup{
  pdftitle={Johns Hopkins Coursera - Statistical Inference - Project Part 2},
  pdfauthor={shostiou},
  hidelinks,
  pdfcreator={LaTeX via pandoc}}
\urlstyle{same} % disable monospaced font for URLs
\usepackage[margin=1in]{geometry}
\usepackage{color}
\usepackage{fancyvrb}
\newcommand{\VerbBar}{|}
\newcommand{\VERB}{\Verb[commandchars=\\\{\}]}
\DefineVerbatimEnvironment{Highlighting}{Verbatim}{commandchars=\\\{\}}
% Add ',fontsize=\small' for more characters per line
\usepackage{framed}
\definecolor{shadecolor}{RGB}{248,248,248}
\newenvironment{Shaded}{\begin{snugshade}}{\end{snugshade}}
\newcommand{\AlertTok}[1]{\textcolor[rgb]{0.94,0.16,0.16}{#1}}
\newcommand{\AnnotationTok}[1]{\textcolor[rgb]{0.56,0.35,0.01}{\textbf{\textit{#1}}}}
\newcommand{\AttributeTok}[1]{\textcolor[rgb]{0.77,0.63,0.00}{#1}}
\newcommand{\BaseNTok}[1]{\textcolor[rgb]{0.00,0.00,0.81}{#1}}
\newcommand{\BuiltInTok}[1]{#1}
\newcommand{\CharTok}[1]{\textcolor[rgb]{0.31,0.60,0.02}{#1}}
\newcommand{\CommentTok}[1]{\textcolor[rgb]{0.56,0.35,0.01}{\textit{#1}}}
\newcommand{\CommentVarTok}[1]{\textcolor[rgb]{0.56,0.35,0.01}{\textbf{\textit{#1}}}}
\newcommand{\ConstantTok}[1]{\textcolor[rgb]{0.00,0.00,0.00}{#1}}
\newcommand{\ControlFlowTok}[1]{\textcolor[rgb]{0.13,0.29,0.53}{\textbf{#1}}}
\newcommand{\DataTypeTok}[1]{\textcolor[rgb]{0.13,0.29,0.53}{#1}}
\newcommand{\DecValTok}[1]{\textcolor[rgb]{0.00,0.00,0.81}{#1}}
\newcommand{\DocumentationTok}[1]{\textcolor[rgb]{0.56,0.35,0.01}{\textbf{\textit{#1}}}}
\newcommand{\ErrorTok}[1]{\textcolor[rgb]{0.64,0.00,0.00}{\textbf{#1}}}
\newcommand{\ExtensionTok}[1]{#1}
\newcommand{\FloatTok}[1]{\textcolor[rgb]{0.00,0.00,0.81}{#1}}
\newcommand{\FunctionTok}[1]{\textcolor[rgb]{0.00,0.00,0.00}{#1}}
\newcommand{\ImportTok}[1]{#1}
\newcommand{\InformationTok}[1]{\textcolor[rgb]{0.56,0.35,0.01}{\textbf{\textit{#1}}}}
\newcommand{\KeywordTok}[1]{\textcolor[rgb]{0.13,0.29,0.53}{\textbf{#1}}}
\newcommand{\NormalTok}[1]{#1}
\newcommand{\OperatorTok}[1]{\textcolor[rgb]{0.81,0.36,0.00}{\textbf{#1}}}
\newcommand{\OtherTok}[1]{\textcolor[rgb]{0.56,0.35,0.01}{#1}}
\newcommand{\PreprocessorTok}[1]{\textcolor[rgb]{0.56,0.35,0.01}{\textit{#1}}}
\newcommand{\RegionMarkerTok}[1]{#1}
\newcommand{\SpecialCharTok}[1]{\textcolor[rgb]{0.00,0.00,0.00}{#1}}
\newcommand{\SpecialStringTok}[1]{\textcolor[rgb]{0.31,0.60,0.02}{#1}}
\newcommand{\StringTok}[1]{\textcolor[rgb]{0.31,0.60,0.02}{#1}}
\newcommand{\VariableTok}[1]{\textcolor[rgb]{0.00,0.00,0.00}{#1}}
\newcommand{\VerbatimStringTok}[1]{\textcolor[rgb]{0.31,0.60,0.02}{#1}}
\newcommand{\WarningTok}[1]{\textcolor[rgb]{0.56,0.35,0.01}{\textbf{\textit{#1}}}}
\usepackage{graphicx,grffile}
\makeatletter
\def\maxwidth{\ifdim\Gin@nat@width>\linewidth\linewidth\else\Gin@nat@width\fi}
\def\maxheight{\ifdim\Gin@nat@height>\textheight\textheight\else\Gin@nat@height\fi}
\makeatother
% Scale images if necessary, so that they will not overflow the page
% margins by default, and it is still possible to overwrite the defaults
% using explicit options in \includegraphics[width, height, ...]{}
\setkeys{Gin}{width=\maxwidth,height=\maxheight,keepaspectratio}
% Set default figure placement to htbp
\makeatletter
\def\fps@figure{htbp}
\makeatother
\setlength{\emergencystretch}{3em} % prevent overfull lines
\providecommand{\tightlist}{%
  \setlength{\itemsep}{0pt}\setlength{\parskip}{0pt}}
\setcounter{secnumdepth}{-\maxdimen} % remove section numbering

\title{Johns Hopkins Coursera - Statistical Inference - Project Part 2}
\author{shostiou}
\date{06/09/2020}

\begin{document}
\maketitle

\hypertarget{part-02---basic-inferential-data-analysis}{%
\subsection{Part 02 - Basic Inferential Data
Analysis}\label{part-02---basic-inferential-data-analysis}}

This second part of the project is dedicated to the analysis of the
ToothGrowth dataset (embedded in the R package).

\hypertarget{data-upload-basic-exploration}{%
\subsubsection{Data upload \& basic
exploration}\label{data-upload-basic-exploration}}

Let's start by loading the dataset and by doing basic explorations.\\
Refering to the help associated to the data, this dataset is describe as
: \textbf{``The response is the length of odontoblasts (cells
responsible for tooth growth) in 60 guinea pigs. Each animal received
one of three dose levels of vitamin C (0.5, 1, and 2 mg/day) by one of
two delivery methods, orange juice or ascorbic acid (a form of vitamin C
and coded as VC).''}

Content :\\
60 observations on 3 variables. {[},1{]} len numeric Tooth length
{[},2{]} supp factor Supplement type (VC or OJ). {[},3{]} dose numeric
Dose in milligrams/day

\begin{Shaded}
\begin{Highlighting}[]
\CommentTok{# Loading the dataset}
\KeywordTok{data}\NormalTok{(ToothGrowth)  }
\CommentTok{# Getting general overview of the data  }
\KeywordTok{str}\NormalTok{(ToothGrowth)  }
\end{Highlighting}
\end{Shaded}

\begin{verbatim}
## 'data.frame':    60 obs. of  3 variables:
##  $ len : num  4.2 11.5 7.3 5.8 6.4 10 11.2 11.2 5.2 7 ...
##  $ supp: Factor w/ 2 levels "OJ","VC": 2 2 2 2 2 2 2 2 2 2 ...
##  $ dose: num  0.5 0.5 0.5 0.5 0.5 0.5 0.5 0.5 0.5 0.5 ...
\end{verbatim}

\begin{Shaded}
\begin{Highlighting}[]
\CommentTok{# Getting summary statistics.}
\KeywordTok{summary}\NormalTok{(ToothGrowth)  }
\end{Highlighting}
\end{Shaded}

\begin{verbatim}
##       len        supp         dose      
##  Min.   : 4.20   OJ:30   Min.   :0.500  
##  1st Qu.:13.07   VC:30   1st Qu.:0.500  
##  Median :19.25           Median :1.000  
##  Mean   :18.81           Mean   :1.167  
##  3rd Qu.:25.27           3rd Qu.:2.000  
##  Max.   :33.90           Max.   :2.000
\end{verbatim}

\begin{Shaded}
\begin{Highlighting}[]
\CommentTok{# Let's visualize the distribution of the numerical values}
\KeywordTok{hist}\NormalTok{(ToothGrowth}\OperatorTok{$}\NormalTok{len, }\DataTypeTok{main =} \StringTok{"distribution of Tooth length"}\NormalTok{)  }
\end{Highlighting}
\end{Shaded}

\includegraphics{JH_M06_W04_Shostiou_Part02_files/figure-latex/data_load-1.pdf}

\begin{Shaded}
\begin{Highlighting}[]
\KeywordTok{hist}\NormalTok{(ToothGrowth}\OperatorTok{$}\NormalTok{dose, }\DataTypeTok{main =} \StringTok{"distribution of Vitamin doses"}\NormalTok{)}
\end{Highlighting}
\end{Shaded}

\includegraphics{JH_M06_W04_Shostiou_Part02_files/figure-latex/data_load-2.pdf}

\begin{Shaded}
\begin{Highlighting}[]
\CommentTok{# Nb of rows}
\NormalTok{df_nb_rows =}\StringTok{ }\KeywordTok{as.numeric}\NormalTok{(}\KeywordTok{nrow}\NormalTok{(ToothGrowth))}
\end{Highlighting}
\end{Shaded}

We can observe that the data contains 2 groups of observations. Each of
those groups is made of 30 observation (Orange Juice / Ascorbic Acid).\\
Even though the vitamin doses is numerical data, the values observed in
the dataset is limited to 3 levels : 0.5; 1 ; 2. We can consider this
variable as a discrete (not coninuous).

Let's pursue the exploration by comparing the distribution of Tooth
length based on the delivery method used (orange Juice - OJ ; ascorbid
acid - VC)

\begin{Shaded}
\begin{Highlighting}[]
\CommentTok{# We call use ggplot2 to show the distributions  }
\KeywordTok{library}\NormalTok{(ggplot2)}
\CommentTok{# Building the histograms based on supp variable  }
\KeywordTok{ggplot}\NormalTok{(ToothGrowth, }\KeywordTok{aes}\NormalTok{(}\DataTypeTok{x =}\NormalTok{ len)) }\OperatorTok{+}
\StringTok{  }\KeywordTok{geom_histogram}\NormalTok{(}\KeywordTok{aes}\NormalTok{(}\DataTypeTok{color =}\NormalTok{ supp, }\DataTypeTok{fill =}\NormalTok{ supp), }
                \DataTypeTok{position =} \StringTok{"identity"}\NormalTok{, }\DataTypeTok{alpha =} \FloatTok{0.4}\NormalTok{) }\OperatorTok{+}
\StringTok{  }\KeywordTok{scale_color_manual}\NormalTok{(}\DataTypeTok{values =} \KeywordTok{c}\NormalTok{(}\StringTok{"#00AFBB"}\NormalTok{, }\StringTok{"#E7B800"}\NormalTok{)) }\OperatorTok{+}
\StringTok{  }\KeywordTok{scale_fill_manual}\NormalTok{(}\DataTypeTok{values =} \KeywordTok{c}\NormalTok{(}\StringTok{"#00AFBB"}\NormalTok{, }\StringTok{"#E7B800"}\NormalTok{))}
\end{Highlighting}
\end{Shaded}

\begin{verbatim}
## `stat_bin()` using `bins = 30`. Pick better value with `binwidth`.
\end{verbatim}

\includegraphics{JH_M06_W04_Shostiou_Part02_files/figure-latex/hist_by_del_method-1.pdf}
The observation of the 2 overlapped distributions doesn't provide a key
evidence between Tooth lengths \& the delivery method used during the
experiment.

\hypertarget{comparing-the-means}{%
\subsubsection{Comparing the means}\label{comparing-the-means}}

Let's compare the means of the tooth length distribution based on the
delivery method.

\begin{Shaded}
\begin{Highlighting}[]
\CommentTok{# Let's call the dplyr package}
\KeywordTok{library}\NormalTok{(dplyr)}
\end{Highlighting}
\end{Shaded}

\begin{verbatim}
## 
## Attaching package: 'dplyr'
\end{verbatim}

\begin{verbatim}
## The following objects are masked from 'package:stats':
## 
##     filter, lag
\end{verbatim}

\begin{verbatim}
## The following objects are masked from 'package:base':
## 
##     intersect, setdiff, setequal, union
\end{verbatim}

\begin{Shaded}
\begin{Highlighting}[]
\CommentTok{# means & standard error computation  }
\NormalTok{ToothGrowth }\OperatorTok\StringTok{ }\KeywordTok{group_by}\NormalTok{(supp) }\OperatorTok\StringTok{ }\KeywordTok{summarize}\NormalTok{(}\DataTypeTok{mean_len =} \KeywordTok{mean}\NormalTok{(len), }\DataTypeTok{std_len =} \KeywordTok{sd}\NormalTok{(len))}
\end{Highlighting}
\end{Shaded}

\begin{verbatim}
## `summarise()` ungrouping output (override with `.groups` argument)
\end{verbatim}

\begin{verbatim}
## Warning: `...` is not empty.
## 
## We detected these problematic arguments:
## * `needs_dots`
## 
## These dots only exist to allow future extensions and should be empty.
## Did you misspecify an argument?
\end{verbatim}

\begin{verbatim}
## # A tibble: 2 x 3
##   supp  mean_len std_len
##   <fct>    <dbl>   <dbl>
## 1 OJ        20.7    6.61
## 2 VC        17.0    8.27
\end{verbatim}

We can observe that mean values of the 2 distributions (tooth length
with orange juice OJ / tooth length with ascorbic acis CV) differs.

\hypertarget{hypothesis-test}{%
\subsubsection{Hypothesis Test}\label{hypothesis-test}}

Now to determine is this difference is statistically pertinent, we will
use a Hypothesis test : H0 : xbar\_len\_OJ - xbar\_len\_CV = 0 HA :
xbar\_len\_OJ - xbar\_len\_CV != 0\\
Note that the 2 groups will be considered as being unpaired.\\
As a second assumption, we will consider a constant variance in the
population.

Finally we will conclude the hypothesis test by computing the p-value
associated to a confidence interval of 95\%. The test to be used will be
2 sided test. Note : a two tail test will be used.\\
Let's use T distribution to calculate the p-value.

\begin{Shaded}
\begin{Highlighting}[]
\CommentTok{# Let's define specific dataframes for OJ and VC}
\NormalTok{len_OJ <-}\StringTok{ }\NormalTok{ToothGrowth }\OperatorTok\StringTok{ }\KeywordTok{filter}\NormalTok{(supp }\OperatorTok{==}\StringTok{ 'OJ'}\NormalTok{) }\OperatorTok\StringTok{ }\KeywordTok{select}\NormalTok{(len)}
\NormalTok{len_VC <-}\StringTok{ }\NormalTok{ToothGrowth }\OperatorTok\StringTok{ }\KeywordTok{filter}\NormalTok{(supp }\OperatorTok{==}\StringTok{ 'VC'}\NormalTok{) }\OperatorTok\StringTok{ }\KeywordTok{select}\NormalTok{(len)}
\CommentTok{# Applying the T test using the R native command}
\KeywordTok{t.test}\NormalTok{(len_OJ,len_VC, }\DataTypeTok{alternative=}\StringTok{"two.sided"}\NormalTok{, }\DataTypeTok{paired=}\OtherTok{FALSE}\NormalTok{, }\DataTypeTok{mu =} \DecValTok{0}\NormalTok{, }\DataTypeTok{conf.level =} \FloatTok{0.95}\NormalTok{)}
\end{Highlighting}
\end{Shaded}

\begin{verbatim}
## 
##  Welch Two Sample t-test
## 
## data:  len_OJ and len_VC
## t = 1.9153, df = 55.309, p-value = 0.06063
## alternative hypothesis: true difference in means is not equal to 0
## 95 percent confidence interval:
##  -0.1710156  7.5710156
## sample estimates:
## mean of x mean of y 
##  20.66333  16.96333
\end{verbatim}

\textbf{CONCLUSION :} as p-value \textgreater{} 0.05 we failed to reject
the null hypothesis.\\
The data doesn't provide evidence of differences in means between the
Orange Juice and Ascorbic Acid feeding methods.

\end{document}
