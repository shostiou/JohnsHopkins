% Options for packages loaded elsewhere
\PassOptionsToPackage{unicode}{hyperref}
\PassOptionsToPackage{hyphens}{url}
%
\documentclass[
]{article}
\usepackage{lmodern}
\usepackage{amssymb,amsmath}
\usepackage{ifxetex,ifluatex}
\ifnum 0\ifxetex 1\fi\ifluatex 1\fi=0 % if pdftex
  \usepackage[T1]{fontenc}
  \usepackage[utf8]{inputenc}
  \usepackage{textcomp} % provide euro and other symbols
\else % if luatex or xetex
  \usepackage{unicode-math}
  \defaultfontfeatures{Scale=MatchLowercase}
  \defaultfontfeatures[\rmfamily]{Ligatures=TeX,Scale=1}
\fi
% Use upquote if available, for straight quotes in verbatim environments
\IfFileExists{upquote.sty}{\usepackage{upquote}}{}
\IfFileExists{microtype.sty}{% use microtype if available
  \usepackage[]{microtype}
  \UseMicrotypeSet[protrusion]{basicmath} % disable protrusion for tt fonts
}{}
\makeatletter
\@ifundefined{KOMAClassName}{% if non-KOMA class
  \IfFileExists{parskip.sty}{%
    \usepackage{parskip}
  }{% else
    \setlength{\parindent}{0pt}
    \setlength{\parskip}{6pt plus 2pt minus 1pt}}
}{% if KOMA class
  \KOMAoptions{parskip=half}}
\makeatother
\usepackage{xcolor}
\IfFileExists{xurl.sty}{\usepackage{xurl}}{} % add URL line breaks if available
\IfFileExists{bookmark.sty}{\usepackage{bookmark}}{\usepackage{hyperref}}
\hypersetup{
  pdftitle={JH - M07 - Regression Project},
  pdfauthor={S. Hostiou},
  hidelinks,
  pdfcreator={LaTeX via pandoc}}
\urlstyle{same} % disable monospaced font for URLs
\usepackage[margin=1in]{geometry}
\usepackage{color}
\usepackage{fancyvrb}
\newcommand{\VerbBar}{|}
\newcommand{\VERB}{\Verb[commandchars=\\\{\}]}
\DefineVerbatimEnvironment{Highlighting}{Verbatim}{commandchars=\\\{\}}
% Add ',fontsize=\small' for more characters per line
\usepackage{framed}
\definecolor{shadecolor}{RGB}{248,248,248}
\newenvironment{Shaded}{\begin{snugshade}}{\end{snugshade}}
\newcommand{\AlertTok}[1]{\textcolor[rgb]{0.94,0.16,0.16}{#1}}
\newcommand{\AnnotationTok}[1]{\textcolor[rgb]{0.56,0.35,0.01}{\textbf{\textit{#1}}}}
\newcommand{\AttributeTok}[1]{\textcolor[rgb]{0.77,0.63,0.00}{#1}}
\newcommand{\BaseNTok}[1]{\textcolor[rgb]{0.00,0.00,0.81}{#1}}
\newcommand{\BuiltInTok}[1]{#1}
\newcommand{\CharTok}[1]{\textcolor[rgb]{0.31,0.60,0.02}{#1}}
\newcommand{\CommentTok}[1]{\textcolor[rgb]{0.56,0.35,0.01}{\textit{#1}}}
\newcommand{\CommentVarTok}[1]{\textcolor[rgb]{0.56,0.35,0.01}{\textbf{\textit{#1}}}}
\newcommand{\ConstantTok}[1]{\textcolor[rgb]{0.00,0.00,0.00}{#1}}
\newcommand{\ControlFlowTok}[1]{\textcolor[rgb]{0.13,0.29,0.53}{\textbf{#1}}}
\newcommand{\DataTypeTok}[1]{\textcolor[rgb]{0.13,0.29,0.53}{#1}}
\newcommand{\DecValTok}[1]{\textcolor[rgb]{0.00,0.00,0.81}{#1}}
\newcommand{\DocumentationTok}[1]{\textcolor[rgb]{0.56,0.35,0.01}{\textbf{\textit{#1}}}}
\newcommand{\ErrorTok}[1]{\textcolor[rgb]{0.64,0.00,0.00}{\textbf{#1}}}
\newcommand{\ExtensionTok}[1]{#1}
\newcommand{\FloatTok}[1]{\textcolor[rgb]{0.00,0.00,0.81}{#1}}
\newcommand{\FunctionTok}[1]{\textcolor[rgb]{0.00,0.00,0.00}{#1}}
\newcommand{\ImportTok}[1]{#1}
\newcommand{\InformationTok}[1]{\textcolor[rgb]{0.56,0.35,0.01}{\textbf{\textit{#1}}}}
\newcommand{\KeywordTok}[1]{\textcolor[rgb]{0.13,0.29,0.53}{\textbf{#1}}}
\newcommand{\NormalTok}[1]{#1}
\newcommand{\OperatorTok}[1]{\textcolor[rgb]{0.81,0.36,0.00}{\textbf{#1}}}
\newcommand{\OtherTok}[1]{\textcolor[rgb]{0.56,0.35,0.01}{#1}}
\newcommand{\PreprocessorTok}[1]{\textcolor[rgb]{0.56,0.35,0.01}{\textit{#1}}}
\newcommand{\RegionMarkerTok}[1]{#1}
\newcommand{\SpecialCharTok}[1]{\textcolor[rgb]{0.00,0.00,0.00}{#1}}
\newcommand{\SpecialStringTok}[1]{\textcolor[rgb]{0.31,0.60,0.02}{#1}}
\newcommand{\StringTok}[1]{\textcolor[rgb]{0.31,0.60,0.02}{#1}}
\newcommand{\VariableTok}[1]{\textcolor[rgb]{0.00,0.00,0.00}{#1}}
\newcommand{\VerbatimStringTok}[1]{\textcolor[rgb]{0.31,0.60,0.02}{#1}}
\newcommand{\WarningTok}[1]{\textcolor[rgb]{0.56,0.35,0.01}{\textbf{\textit{#1}}}}
\usepackage{graphicx,grffile}
\makeatletter
\def\maxwidth{\ifdim\Gin@nat@width>\linewidth\linewidth\else\Gin@nat@width\fi}
\def\maxheight{\ifdim\Gin@nat@height>\textheight\textheight\else\Gin@nat@height\fi}
\makeatother
% Scale images if necessary, so that they will not overflow the page
% margins by default, and it is still possible to overwrite the defaults
% using explicit options in \includegraphics[width, height, ...]{}
\setkeys{Gin}{width=\maxwidth,height=\maxheight,keepaspectratio}
% Set default figure placement to htbp
\makeatletter
\def\fps@figure{htbp}
\makeatother
\setlength{\emergencystretch}{3em} % prevent overfull lines
\providecommand{\tightlist}{%
  \setlength{\itemsep}{0pt}\setlength{\parskip}{0pt}}
\setcounter{secnumdepth}{-\maxdimen} % remove section numbering

\title{JH - M07 - Regression Project}
\author{S. Hostiou}
\date{21/10/2020}

\begin{document}
\maketitle

\hypertarget{summary}{%
\subsection{Summary}\label{summary}}

You work for Motor Trend, a magazine about the automobile industry.
Looking at a data set of a collection of cars, they are interested in
exploring the relationship between a set of variables and miles per
gallon (MPG) (outcome). They are particularly interested in the
following two questions:\\
~\\
- Is an automatic or manual transmission better for MPG\\
- Quantify the MPG difference between automatic and manual
transmissions\\

This work is part of the Regression Models course provided by Johns
Hopkins University at Coursera.The ``mtcars'' data set is used for the
purpose of this exercise.\\

The approach followed to answer our 2 questions will be to build a
linear model with the mpg variable as a response variable and a set of
regressors as inputs.\\
This model will provide us elements of interpretation to understand how
the different regressors kept in our model can influence our miles per
gallon concern.\\

\hypertarget{checking-data-set-content}{%
\subsection{Checking Data Set Content}\label{checking-data-set-content}}

Let's start by importing the data into the R environment and by having a
look at the content of the dataframe.

\begin{Shaded}
\begin{Highlighting}[]
\KeywordTok{data}\NormalTok{(}\StringTok{"mtcars"}\NormalTok{)}
\KeywordTok{head}\NormalTok{(mtcars)}
\end{Highlighting}
\end{Shaded}

\begin{verbatim}
##                    mpg cyl disp  hp drat    wt  qsec vs am gear carb
## Mazda RX4         21.0   6  160 110 3.90 2.620 16.46  0  1    4    4
## Mazda RX4 Wag     21.0   6  160 110 3.90 2.875 17.02  0  1    4    4
## Datsun 710        22.8   4  108  93 3.85 2.320 18.61  1  1    4    1
## Hornet 4 Drive    21.4   6  258 110 3.08 3.215 19.44  1  0    3    1
## Hornet Sportabout 18.7   8  360 175 3.15 3.440 17.02  0  0    3    2
## Valiant           18.1   6  225 105 2.76 3.460 20.22  1  0    3    1
\end{verbatim}

The mpg (miles per gallon) variable will be our response variable. Our
approach will be to configure a linear model in order to identify how
the other variables can influence fuel consumption.\\
The following variables will be consisered as numeric continuous : disp
(displacement cu in), hp (horse power),drat (rear axle ratio), wt
(weight), qsec (1/4 mile time).\\
The remaining variables will be handled as factor variables : cyl (nb of
cyl), vs (engine shape), am (auto/man transmission), gear (nb gears),
carb (nb of carb).\\

\begin{Shaded}
\begin{Highlighting}[]
\CommentTok{# for convenience, We will work with a copy of the original data set}
\NormalTok{my_mtcars <-}\StringTok{ }\NormalTok{mtcars}
\end{Highlighting}
\end{Shaded}

\hypertarget{first-assumption}{%
\subsubsection{First assumption}\label{first-assumption}}

The first exploratory plots given in appendix 1 gives us some primary
elements to answer the question. Box plot \& distribution diagrams of
mpg vs transmission mode (auto / man) tend to show lower mpg with
automatic transmission and higher mpg with manual transmission.\\

\hypertarget{collinearity-identification}{%
\subsubsection{Collinearity
Identification}\label{collinearity-identification}}

As a first step we will identify if there are strong evidences of high
correlations between what would be our input variables.

\begin{Shaded}
\begin{Highlighting}[]
\NormalTok{mtcars }\OperatorTok\StringTok{ }\KeywordTok{cor}\NormalTok{()}
\end{Highlighting}
\end{Shaded}

\begin{verbatim}
##             mpg        cyl       disp         hp        drat         wt
## mpg   1.0000000 -0.8521620 -0.8475514 -0.7761684  0.68117191 -0.8676594
## cyl  -0.8521620  1.0000000  0.9020329  0.8324475 -0.69993811  0.7824958
## disp -0.8475514  0.9020329  1.0000000  0.7909486 -0.71021393  0.8879799
## hp   -0.7761684  0.8324475  0.7909486  1.0000000 -0.44875912  0.6587479
## drat  0.6811719 -0.6999381 -0.7102139 -0.4487591  1.00000000 -0.7124406
## wt   -0.8676594  0.7824958  0.8879799  0.6587479 -0.71244065  1.0000000
## qsec  0.4186840 -0.5912421 -0.4336979 -0.7082234  0.09120476 -0.1747159
## vs    0.6640389 -0.8108118 -0.7104159 -0.7230967  0.44027846 -0.5549157
## am    0.5998324 -0.5226070 -0.5912270 -0.2432043  0.71271113 -0.6924953
## gear  0.4802848 -0.4926866 -0.5555692 -0.1257043  0.69961013 -0.5832870
## carb -0.5509251  0.5269883  0.3949769  0.7498125 -0.09078980  0.4276059
##             qsec         vs          am       gear        carb
## mpg   0.41868403  0.6640389  0.59983243  0.4802848 -0.55092507
## cyl  -0.59124207 -0.8108118 -0.52260705 -0.4926866  0.52698829
## disp -0.43369788 -0.7104159 -0.59122704 -0.5555692  0.39497686
## hp   -0.70822339 -0.7230967 -0.24320426 -0.1257043  0.74981247
## drat  0.09120476  0.4402785  0.71271113  0.6996101 -0.09078980
## wt   -0.17471588 -0.5549157 -0.69249526 -0.5832870  0.42760594
## qsec  1.00000000  0.7445354 -0.22986086 -0.2126822 -0.65624923
## vs    0.74453544  1.0000000  0.16834512  0.2060233 -0.56960714
## am   -0.22986086  0.1683451  1.00000000  0.7940588  0.05753435
## gear -0.21268223  0.2060233  0.79405876  1.0000000  0.27407284
## carb -0.65624923 -0.5696071  0.05753435  0.2740728  1.00000000
\end{verbatim}

This correlation matrix allows us to identify strong correlations
between the following variables : cyl \& disp, cyl \& hp, cyl \& vs,
disp \& wt. With the support of the plot of appendix 1 (pair plots
including those 4 variables), decision is taken to suppress the cyl \&
disp variables as they are highly correlated with hp, ws and vs.~This
makes sense because bigger cars will have bigger engines with more
cylinders\ldots{} The other variables will be kept because they can
contribute to explain variability in the model.\\
In addition, moderate correlation values between our response variable
and qsec, gear and carb, we will not consider those variables as well.\\

\hypertarget{model-definition}{%
\subsubsection{Model definition}\label{model-definition}}

In this step, we will fit a set of models by conducting a forward
approach (step by step nested approach with new variables added at each
step). mpg will be our response variable and as our questions are
related to the impact of transmission mode, the am variable will be
embedded in each model. Our categorical variables will be handled as
factors. Anova will be applied to select the best model (nested
likelihood ratio tests).\\

\begin{Shaded}
\begin{Highlighting}[]
\NormalTok{fit0 <-}\StringTok{ }\KeywordTok{lm}\NormalTok{(mpg}\OperatorTok{~}\KeywordTok{factor}\NormalTok{(am),my_mtcars)}
\NormalTok{fit1 <-}\StringTok{ }\KeywordTok{lm}\NormalTok{(mpg}\OperatorTok{~}\KeywordTok{factor}\NormalTok{(am)}\OperatorTok{+}\NormalTok{hp,my_mtcars)}
\NormalTok{fit2 <-}\StringTok{ }\KeywordTok{lm}\NormalTok{(mpg}\OperatorTok{~}\KeywordTok{factor}\NormalTok{(am)}\OperatorTok{+}\NormalTok{drat,my_mtcars)}
\NormalTok{fit3 <-}\StringTok{ }\KeywordTok{lm}\NormalTok{(mpg}\OperatorTok{~}\KeywordTok{factor}\NormalTok{(am)}\OperatorTok{+}\NormalTok{wt,my_mtcars)}
\NormalTok{fit4 <-}\StringTok{ }\KeywordTok{lm}\NormalTok{(mpg}\OperatorTok{~}\KeywordTok{factor}\NormalTok{(am)}\OperatorTok{+}\KeywordTok{factor}\NormalTok{(vs),my_mtcars)}
\NormalTok{fit5 <-}\StringTok{ }\KeywordTok{lm}\NormalTok{(mpg}\OperatorTok{~}\KeywordTok{factor}\NormalTok{(am)}\OperatorTok{+}\NormalTok{hp}\OperatorTok{+}\KeywordTok{factor}\NormalTok{(vs),my_mtcars)}
\NormalTok{fit6 <-}\StringTok{ }\KeywordTok{lm}\NormalTok{(mpg}\OperatorTok{~}\KeywordTok{factor}\NormalTok{(am)}\OperatorTok{+}\NormalTok{hp}\OperatorTok{+}\KeywordTok{factor}\NormalTok{(vs)}\OperatorTok{+}\NormalTok{drat}\OperatorTok{+}\NormalTok{wt,my_mtcars)}

\KeywordTok{anova}\NormalTok{(fit0,fit5)}
\end{Highlighting}
\end{Shaded}

\begin{verbatim}
## Analysis of Variance Table
## 
## Model 1: mpg ~ factor(am)
## Model 2: mpg ~ factor(am) + hp + factor(vs)
##   Res.Df    RSS Df Sum of Sq     F    Pr(>F)    
## 1     30 720.90                                 
## 2     28 218.88  2    502.02 32.11 5.658e-08 ***
## ---
## Signif. codes:  0 '***' 0.001 '**' 0.01 '*' 0.05 '.' 0.1 ' ' 1
\end{verbatim}

As a second step, we control the evolution of the coefficient associated
to the ``am'' variable.

\begin{Shaded}
\begin{Highlighting}[]
\KeywordTok{rbind}\NormalTok{(}\KeywordTok{summary}\NormalTok{(fit0)}\OperatorTok{$}\NormalTok{coef[}\DecValTok{2}\NormalTok{,], }\KeywordTok{summary}\NormalTok{(fit1)}\OperatorTok{$}\NormalTok{coef[}\DecValTok{2}\NormalTok{,], }\KeywordTok{summary}\NormalTok{(fit2)}\OperatorTok{$}\NormalTok{coef[}\DecValTok{2}\NormalTok{,], }\KeywordTok{summary}\NormalTok{(fit3)}\OperatorTok{$}\NormalTok{coef[}\DecValTok{2}\NormalTok{,], }\KeywordTok{summary}\NormalTok{(fit4)}\OperatorTok{$}\NormalTok{coef[}\DecValTok{2}\NormalTok{,], }\KeywordTok{summary}\NormalTok{(fit5)}\OperatorTok{$}\NormalTok{coef[}\DecValTok{2}\NormalTok{,],}\KeywordTok{summary}\NormalTok{(fit6)}\OperatorTok{$}\NormalTok{coef[}\DecValTok{2}\NormalTok{,])}
\end{Highlighting}
\end{Shaded}

\begin{verbatim}
##         Estimate Std. Error     t value     Pr(>|t|)
## [1,]  7.24493927   1.764422  4.10612698 2.850207e-04
## [2,]  5.27708531   1.079541  4.88826953 3.460318e-05
## [3,]  2.80706095   2.282159  1.23000231 2.285814e-01
## [4,] -0.02361522   1.545645 -0.01527855 9.879146e-01
## [5,]  6.06666667   1.274842  4.75875870 4.958115e-05
## [6,]  5.29853680   1.037569  5.10668299 2.071740e-05
## [7,]  2.08595742   1.604920  1.29972713 2.051005e-01
\end{verbatim}

Finally we select fit5 as our best model as this model is the one which
minimizes the std. Error for the coefficient associated to the am
variable. Diagnosis elements associated to the model are given in
appendix number 3 and allow us to conclude that the model fit appears to
be valid.\\

\hypertarget{interpretations}{%
\subsubsection{Interpretations}\label{interpretations}}

On average, with other variables being fixed, compared to an automatic
transmission, manual transmission increases mpg by 5.29. This value is
statistically significant as the Pr associated to this coefficient is
\textless{} to 0.05.

The confidence interval associated to the coefficient (manual
transmission) is estimated below :

\begin{Shaded}
\begin{Highlighting}[]
\KeywordTok{confint}\NormalTok{(fit5)[}\DecValTok{2}\NormalTok{,]}
\end{Highlighting}
\end{Shaded}

\begin{verbatim}
##    2.5 %   97.5 % 
## 3.173173 7.423901
\end{verbatim}

As this confidence interval doesn't include zero, We can conclude that
manual transmission mode is more efficient than automatic transmission
!\\
~\\
~\\

\hypertarget{appendixes---plots}{%
\subsection{Appendixes - Plots}\label{appendixes---plots}}

\hypertarget{appendix-1---eda-mpg-auto-manu-transmission}{%
\subsubsection{Appendix 1 - EDA mpg \textasciitilde{} auto / manu
transmission}\label{appendix-1---eda-mpg-auto-manu-transmission}}

The main concern is related to identification of links between
transmission mode (auto / manu) and mpg. This plot is used to visualize
mpg distributions vs transmission categories.

\begin{Shaded}
\begin{Highlighting}[]
\CommentTok{#ggplot boxplot not printed correctly with Knitr, using basing plotting instead.}
\KeywordTok{boxplot}\NormalTok{(mpg }\OperatorTok{~}\StringTok{ }\NormalTok{am, }\DataTypeTok{data =}\NormalTok{ my_mtcars, }\DataTypeTok{col=}\NormalTok{(}\KeywordTok{c}\NormalTok{(}\StringTok{"red"}\NormalTok{,}\StringTok{"blue"}\NormalTok{)))}
\end{Highlighting}
\end{Shaded}

\includegraphics{JH---M07---shostiou-final_files/figure-latex/unnamed-chunk-7-1.pdf}

\hypertarget{appendix-2---pair-plots}{%
\subsubsection{Appendix 2 - pair plots}\label{appendix-2---pair-plots}}

\begin{Shaded}
\begin{Highlighting}[]
\CommentTok{# Function used to dispkay correlation coefficients on pair plots  }
\NormalTok{panel.cor <-}\StringTok{ }\ControlFlowTok{function}\NormalTok{(x, y, ...)}
\NormalTok{\{}
\KeywordTok{par}\NormalTok{(}\DataTypeTok{usr =} \KeywordTok{c}\NormalTok{(}\DecValTok{0}\NormalTok{, }\DecValTok{1}\NormalTok{, }\DecValTok{0}\NormalTok{, }\DecValTok{1}\NormalTok{))}
\NormalTok{txt <-}\StringTok{ }\KeywordTok{as.character}\NormalTok{(}\KeywordTok{format}\NormalTok{(}\KeywordTok{cor}\NormalTok{(x, y), }\DataTypeTok{digits=}\DecValTok{2}\NormalTok{))}
\KeywordTok{text}\NormalTok{(}\FloatTok{0.5}\NormalTok{, }\FloatTok{0.5}\NormalTok{, txt, }\DataTypeTok{cex =} \DecValTok{3}\OperatorTok{*}\StringTok{ }\KeywordTok{abs}\NormalTok{(}\KeywordTok{cor}\NormalTok{(x, y)))}
\NormalTok{\}}

\NormalTok{my_mtcars }\OperatorTok\StringTok{ }\KeywordTok{select}\NormalTok{ (cyl,disp,hp,wt,vs) }\OperatorTok\StringTok{ }\KeywordTok{pairs}\NormalTok{(}\DataTypeTok{lower.panel=}\NormalTok{panel.cor,}\DataTypeTok{pch =} \DecValTok{19}\NormalTok{)}
\end{Highlighting}
\end{Shaded}

\includegraphics{JH---M07---shostiou-final_files/figure-latex/unnamed-chunk-8-1.pdf}

\hypertarget{appendix-3---model-diagnosis}{%
\subsubsection{Appendix 3 - Model
Diagnosis}\label{appendix-3---model-diagnosis}}

As we selected model fit5 as our best model, let's perform some
diagnosis plots.

\hypertarget{normality-condition-for-the-residuals}{%
\paragraph{Normality condition for the
residuals}\label{normality-condition-for-the-residuals}}

\begin{Shaded}
\begin{Highlighting}[]
\KeywordTok{qqnorm}\NormalTok{(fit5}\OperatorTok{$}\NormalTok{residuals)}
\KeywordTok{qqline}\NormalTok{(fit5}\OperatorTok{$}\NormalTok{residuals)}
\end{Highlighting}
\end{Shaded}

\includegraphics{JH---M07---shostiou-final_files/figure-latex/unnamed-chunk-9-1.pdf}

The normality condition is satisfied.\\

\begin{Shaded}
\begin{Highlighting}[]
\CommentTok{# Valeurs absolues  }
\KeywordTok{plot}\NormalTok{((fit5}\OperatorTok{$}\NormalTok{residuals) }\OperatorTok{~}\StringTok{ }\NormalTok{fit5}\OperatorTok{$}\NormalTok{fitted, }\DataTypeTok{xlab =}\StringTok{"fitted data"}\NormalTok{, }\DataTypeTok{ylab=}\StringTok{"residuals"}\NormalTok{)}
\end{Highlighting}
\end{Shaded}

\includegraphics{JH---M07---shostiou-final_files/figure-latex/unnamed-chunk-10-1.pdf}

The constant variability of the residuals versus the fitted data
condition is satisfied.

\end{document}
